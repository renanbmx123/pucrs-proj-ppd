\documentclass{article}

\usepackage[utf8]{inputenc}
\usepackage[english]{babel}
\usepackage{blindtext}
\usepackage{graphicx}
\usepackage{amsmath}
\usepackage{float}
\usepackage[compact]{titlesec}
\usepackage{multicol}
\usepackage[a4paper, total={7.5in, 10in}]{geometry}

\setlength{\columnsep}{1cm}
\setlength{\parindent}{0em}
\titlespacing{\section}{0pt}{*0}{*0}

\begin{document}

\begin{multicols*}{2}
\textbf{Relatório de Entrega de Trabalho} \newline
\textbf{Disciplina de programação Paralela (PP)}\newline 
\textbf{Prof. César De Rose} \newline
\textbf{Alunos:} Rafael Rios e Rodrigo Silveira \newline
\textbf{Exercício:} TPD1: Modelo Cliente/Servidor (RPC) \newline

\section{Implementação}

A fim de avaliar a viabilidade de balanceamento de carga em um
ambiente sem disponibilidade de alocação dinâmica de nodos, como
no caso do LAD, implementou-se um codigo que busca balancear a
carga de trabalho entre todos os nodos alocados manualmente para
o MPI. O critério utilizado foi o seguinte: cada nodo deve processar
quantidade semelhante de trabalho em relação aos demais nodos,
independentemente da quantidade de processos executados
localmente. A partir da alocação manual, a quantidade de processos
executados em cada nodo pode variar, pois o MPI os distribui entre
os nodos alocados através de um algoritmo semelhante a um Round
Robin. Desta forma, processos alocados em nodos com uma maior
quantidade de processos receberão unidades de trabalho
proporcionalmente menores do que processos que executem em
nodos com menor quantidade de processos alocados. O algoritmo
implementado foi um mergesort, também implementado em versão
sequencial (não paralela) para referência de speed up. A figura 1
mostra um mapa de execução de exemplo, considerando-se um
vetor com N = 500, 5 nodos alocados e 7 processos paralelos
disparados.

\begin{figure}[H]
    \centering
    \includegraphics[width=4cm, height=4cm]{lion.png}
    \caption{Leão fofo}
\end{figure}

Conforme mostra a figura 1, o processo raiz (rank = 0) inicia com
uma unidade de trabalho contendo 500 elementos, sendo que
processará 50 localmente, repassando cerca de 250 elementos para
o processo 1 e mais 200 para o processo 2. Por sua vez, o processo 1,
ao receber os 250 elementos, processará 50 elementos localmente,
repassando o restante para seus filhos, e assim por diante. Porém,
diferentemente de um algoritmo de D\&C clássico, onde a
quantidade de trabalho processada localmente (conquistada)
depende de um valor delta pré-configurado, nesta abordagem a
quantidade absoluta de elementos a ser conquistada dependerá do
valor indicado no mapa de execução para cada processo, assim
como a quantidade de elementos a ser repassada a cada filho
imediato. Este mapa de execução, por sua vez, é enviado pelo
processo raiz a todos os demais processos trabalhadores por uma
operação de broadcast após uma fase de escuta (onde a raiz recebe
de todos os processos trabalhadores seus ranks e identificadores
dos nodos onde estão sendo executados) e montagem do mapa de
execução. 

\section{Dificuldades encontradas}
Depuração do código MPI; alta complexidade dos testes.

\section{Testes}
Foram executados testes com alocação de 2 nodos no cluster Gates,
com execução do código sequencial e do código paralelo com 3, 7 e
15 processos paralelos. A quantidade de processos ímpar foi
utilizada a fim de manter a árvore de processos sempre simétrica. A
quantidade de trabalho inicial para todos os cenários testados foi
fixada em 220 elementos ordenados decrescentemente, de forma a
gerar sempre o pior caso para o mergesort.

\section{Análise do Desempenho}
A versão paralela do mergesort obteve um desempenho superior a
versão sequencial especialmente quando o número de processos é
pequeno, conforme mostrado na figura 2. No entanto, o
desempenho degrada a medida em que a quantidade de processos é
aumentada. Provavelmente, este comportamento se deve ao
overhead gerado pela fase de escuta e pelo processamento e envio
do mapa de execução, que é centralizado no processo raiz e gera
aumento significativo na quantidade de mensagens trocadas, além
de provavelmente gerar um pequeno delay aos trabalhadores logo
no início do processamento.

\section{Observações Finais}
O balanceamento de carga se mostrou viável para o algoritmo de
mergesort implementado na forma de D\&C em um ambiente sem
alocação dinâmica de recursos. A degradação do desempenho
verificada com o aumento da quantidade de processos pode ser
mitigada limitando-se o lançamento de um processo por núcleo
disponível por nodo, conforme sugerem os resultados. Ainda, este
desempenho provavelmente poderá ser melhorado ao se utilizarem
de estruturas não bloqueantes de comunicação MPI.
Fontes no LAD: $/home/filipo.mor/PP/balanced_mergeso$

\end{multicols*}
\end{document}
